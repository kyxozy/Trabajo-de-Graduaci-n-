%%
%%	AVISO IMPORTANTE
%%	Formato optimizado para el sistema operativo GNU/Linux 64 bits
%%	usar TexLive 2016 (o superior), http://www.ctan.org/tex-archive/systems/texlive/Images/
%%	usar TeXstudio 2.11, http://texstudio.sourceforge.net/

\documentclass[letterpaper,12pt]{thesisECFM}
\usepackage{macros}

%%	NO OLVIDE INCLUIR FUENTE DE LAS TABLAS Y FIGURAS

% Decomentar para anular recuadros en los hiperenlaces dentro del pdf
% \hypersetup{pdfborder={0 0 0}}

% Teoremas ---------------------------------------------------------
% estos ambientes son para teoremas, lemas, corolarios, otros
% si no los utiliza los puede obviar en su trabajo de graduación
\theoremstyle{plain}
\newtheorem{thm}{Teorema}[section]
\newtheorem{cor}{Corolario}[chapter]
\newtheorem{lem}{Lema}[chapter]
\newtheorem{prp}{Proposición}[chapter]

\theoremstyle{definition}
\newtheorem{exa}{Ejemplo}[chapter]
\newtheorem{defn}{Definición}[chapter]
\newtheorem{axm}{Axioma}[chapter]

\theoremstyle{remark}
\newtheorem{rem}{Nota}[chapter]

% Operadores y funciones -------------------------------------------
% Los siguientes son ejemplos de comandos definidos por el usuario
% puede borrarlos, únicamente están para mostrar cómo se construyen con LaTeX
\DeclareMathOperator{\Supp}{Supp}       \DeclareMathOperator{\vol}{Vol}%
\DeclareMathOperator{\Rz}{Re}           \DeclareMathOperator{\Iz}{Im}%

\newcommand{\R}{\mathbb{R}}             \newcommand{\Z}{\mathbb{Z}}%
\newcommand{\C}{\mathbb{C}}             \newcommand{\K}{\mathbb{K}}%
\newcommand{\N}{\mathbb{N}}             \newcommand{\Q}{\mathbb{Q}}%
\newcommand{\Af}{\mathfrak{A}}          \newcommand{\Bf}{\mathfrak{B}}%
\newcommand{\Cf}{\mathfrak{C}}          \newcommand{\Df}{\mathfrak{D}}%
\newcommand{\Ff}{\mathfrak{F}}          \newcommand{\Lf}{\mathfrak{L}}%
\newcommand{\Mf}{\mathfrak{M}}          \newcommand{\Sf}{\mathfrak{S}}%
\newcommand{\Hi}{\mathcal{H}}           \newcommand{\Ba}{\mathcal{B}}%
\newcommand{\nada}{\varnothing}         \newcommand{\To}{\longrightarrow}%
\newcommand{\RR}{[-\infty,+\infty]}     \newcommand{\df}{:=}%
\newcommand{\sani}{$\sigma$\nobreakdash-anillo}
\newcommand{\salg}{$\sigma$\nobreakdash-álgebra}
\newcommand{\ff}{f^{-1}}

\newcommand{\norm}[1]{\left\Vert#1\right\Vert}
\newcommand{\pnorm}[2]{\norm{#1}_{#2}}
\newcommand{\abs}[1]{\left\vert#1\right\vert}
\newcommand{\su}[2]{\left\{#1_{#2}\right\}}
\newcommand{\Su}[4]{\suce #1#2 _{#2=#3}^{#4}}
\newcommand{\union}[4]{\bigcup_{{#2}={#3}}^{#4} #1_{#2}}
\newcommand{\Union}[4]{\bigcup \limits_{{#2}={#3}}^{#4} #1_{#2}}
\newcommand{\inter}[4]{\bigcap_{{#2}={#3}}^{#4} #1_{#2}}
\newcommand{\Inter}[4]{\bigcap \limits_{{#2}={#3}}^{#4} #1_{#2}}

\newcommand{\pdual}[2]{\left<#1,#2\right>}
\newcommand{\bra}[1]{\left<#1\right\vert}
\newcommand{\ket}[1]{\left\vert#1\right>}
\newcommand{\braket}[2]{\left<#1\vert#2\right>}
\newcommand{\Braket}[2]{\left\vert#1\right>\! \left<#2\right\vert}
%%%%%%%%%%%%%%%%%%%%%%%%%%%%%%%%%%%%%%%%%%%%%%%%%%%%%%%%%%%%%%%%%%%%


% Cuerpo de la tesis -----------------------------------------------

\begin{document}

%% Datos generales del trabajo de graduación
\datosThesis%
{2}%						% física 1; matemática 2
{El teorema de clasificación de superficies compactas}%		% Título del trabajo de graduación
{Lourdes Kristel Rosales Alarcón}%			% autor
{Alan Reyes}%			% asesor
{julio de 2017}		% mes y año de la orden de impresión
{2}							% femenino 1; masculino 2

%% Datos generales del examen general privado
\examenPrivado%
{M.Sc. Edgar Anibal Cifuentes Anléu}%	% director ECFM
{Ing. José Rodolfo Samayoa Dardón}%		% secretario académico
{...}%		% examinador 1
{...}%		% examinador 2
{...}%		% examinador 3

{\onehalfspacing	% interlineado 1 1/2

\OrdenImpresion{ordenImpresion}		% incluye orden de impresión, guardada en pdf

\Agrade{agradecimientos}			% Agradecimientos



\Dedica{dedicatoria}				% Dedicatoria

\par}
 
\frontmatter    % --------------------------------------------------  Hojas preliminares

{\onehalfspacing	% interlineado 1 1/2

\tableofcontents    % Índice general vinculado

%%% \figurasYtablas{ lista_figuras }{ lista_tablas }; con valor 1 se incluye la lista,
%%% cualquier otro valor no la genera
\figurasYtablas{1}{1}

%%% INCLUYA LA SIMBOLOGÍA NECESARIA EN ESTE APARTADO
%%% NO CAMBIAR LA DEFINICIÓN DE LA TABLA LARGA


\chapter{LISTA DE SÍMBOLOS}

\begin{longtable}{@{}l@{\extracolsep{\fill}} p{4.75in} @{}}  %%%	NO CAMBIAR ESTA LÍNEA
  \textsf{Símbolo} & \textsf{Significado}\\[12pt]
  \endhead
  $:=$ & es definido por\\
  $\cong$ & es isomorfo a\\
  $\Leftrightarrow$ & si y sólo si\\
  $E^c$ & complemento de $E$\\
  $\varsubsetneq$ & estrictamente contenido\\
  $E\setminus F$ & diferencia entre $E$ y $F$\\
  $E\Delta F$ & diferencia simétrica entre $E$ y
  $F$\\
  $\mathcal{P} (X)$ & conjunto potencia de $X$\\
  $\chi_E$ & función característica de $E$\\
  $E_n\!\!\uparrow$ & $E_n$ es una sucesión
  creciente\\
  $\mathfrak{L}$ & \salg{} de los conjuntos
  Lebesgue"=medibles\\
  $\mathscr{S}$ & espacio muestral\\
  $\mathfrak{A}$ & \salg{} de eventos\\
  $(\mathscr{S},\mathfrak{A},P)$ & espacio de
  probabilidad\\
  $\mathscr{D}$ & espacio de las funciones de
  prueba\\
  $\mathscr{D}'$ & espacio de las distribuciones\\
  $\delta_0$ & medida de Dirac, función $\delta$ de Dirac o
  $\delta$-función\\
  $\Phi^{\times}$ & espacio antidual de $\Phi$\\
  $\Phi\subset \mathcal{H}\subset \Phi^{\times}$ &
  espacio de Hilbert equipado o tripleta de Gel'fand\\
  $\left\vert \psi \right>$ & vector \emph{ket}\\
  $\left< \psi \right\vert$ & funcional \emph{bra}\\
  $\left< \varphi \vert \psi \right>$ & \emph{braket}
\end{longtable}
  % Lista de símbolos

%%% Haga el diseño que más le guste
\chapter{OBJETIVOS}

\section*{General}
Escriba el objetivo general.


\section*{Específicos}
Enumere los objetivos específicos.
\begin{enumerate}
\item 
\item 
\end{enumerate}

      % Resumen y objetivos

%%% Haga el diseño que más le guste
\chapter{INTRODUCCIÓN}


      % Introducción

\mainmatter     % --------------------------------------------------  Cuerpo del Trabajo de Graduación

\chapter{CONCEPTOS PRELIMINARES}

% ---------------> conjuntos, funciones y sucesiones
\section{subespacios, espacio producto, espacios de unión disjuntos}
Esta es una nueva linea      % Cap. 1 

%\chapter{SUPERFICIES}
      % Cap. 2 

%\chapter{EL GRUPO FUNDAMENTAL}

      % Cap. 3 

{\backmatter     %	Capítulos no van numerados --------------------------------------------------  Apartados finales

%%% INCLUYA SUS CONCLUSIONES Y RECOMENDACIONES


\chapter{CONCLUSIONES}
\begin{enumerate}
	\item Conclusión 1.
	\item Conclusión 2.
	\item Conclusión 3.
\end{enumerate}

\chapter{RECOMENDACIONES}
\begin{enumerate}
	\item Recomendación 1.
	\item Recomendación 2.
	\item Recomendación 3.
\end{enumerate}
     % Conclusiones y recomendaciones

\begin{thebibliography}{99}
%% La bibliografía se ordena en orden alfabético respecto al apellido del 
%% autor o autor principal
%% cada entrada tiene su formatado dependiendo si es libro, artículo,
%% tesis, contenido en la web, etc

% artículo en una revista arbitrada
\bibitem{albin} P. Albin, E. Leichtnam, R. Mazzeo y P. Piazza. The signature package on Witt spaces. \textit{Ann. Sci. Ec. Norm. Supér. (4)}, \textbf{45}(2):241--310, 2012.

% libro
\bibitem{Brez} H. Brezis. \textit{Analyse functionnelle, théorie et applications.} (Collection Mathématiques Appliquées pour la Maítrise) Masson, Paris, 1992.

% libro
\bibitem{Choq} Y. Choquet"=Bruhat y otros. \textit{Analysis, manifolds and physics. (volumen 1)} North"=Holland Publishing Company, Amsterdam, 1996.

% libro
\bibitem{C:H2} R. Courant y {D. Hilbert}. \textit{Methods of mathematical physics. (volumen 2)} Interscience Publishers, Nueva York, 1962.

% artículo en la web arXiv, preprint
\bibitem{Rafa4} R. {De la Madrid}. The rigged {Hilbert} space of the free hamiltonian. Consultado en marzo de 2005 en \url{http://arxiv.org/abs/quant-ph/0210167}.

% libro con datos faltantes, s.e. "sin editorial"
\bibitem{Doc} J. Escamilla-Castillo. \textit{Topología.} 2"a ed. s.e., Guatemala, 1992.

% libro traducido
\bibitem{Hasr} N. Haaser y {J. Sullivan}. \textit{Análisis real.}  Tr.~Ricardo Vinós. Trillas, México, 1978.

% libro traducido
\bibitem{Halmo} P. Halmos. \textit{Teoría intuitiva de los conjuntos.} 8"a ed. Tr.~Antonio Martín. Compañía Editorial Continental, S.A., México, 1973.

% libro 
\bibitem{Haus} F. Hausdorff. \textit{Set theory.} 2"a ed. Chelsea Publishing Company, Nueva York, 1962.

% libro
\bibitem{Heis} W. Heisenberg. \textit{The physical principles of the quantum theory.} Dover Publications, Inc., Nueva York, 1949.

% libro
\bibitem{Hewit} E. Hewitt y {K. Stromberg}. \textit{Real and abstract analysis.} Springer"=Verlag, Nueva York, 1965.

% libro
\bibitem{Kolmo} A. Kolmogorov y {S. Fomin}. \textit{Elementos de la teoría de funciones y del análisis funcional.} Tr.~Carlos Vega. MIR, Moscú, 1975.

% artículo de una enciclopedia en línea
\bibitem{Kronz} F. Kronz. Quantum theory: von {Neumann} versus {Dirac}. Consultado en marzo de 2005 en \url{http://plato.stanford.edu/entries/qt-nvd/}.

% artículo en una revista arbitrada
\bibitem{liu} K. Liu, X. Sun, and S.-T. Yau. Goodness of canonical metrics on the moduli space of Riemann surfaces. \textit{Pure Appl. Math. Q.}, \textbf{10}(2):223--243, 2014

% artículo en una revista arbitrada
\bibitem{leader} E. Leader and C. Lorcé, The angular momentum controversy: What's it all about and does it matter?, \textit{Phys. Rept.} \textbf{541}, 163 (2014).

% libro
\bibitem{Omnes} Omn\`{e}s, R. \textit{The interpretation of quantum mechanics.} (Princeton Series in Physics) Princeton University Press, Princeton, 1994.

% libro
\bibitem{Penr} R. Penrose. \textit{La mente nueva del emperador.} Tr.~José García. Fondo de Cultura Económica, México, 1996.

% publicación electrónica
\bibitem{Ster} S. Sternberg. Theory of functions of a real variable. Consultado en abril de 2005 en \url{http://www.math.harvard.edu/\~shlomo}.

% publicación electrónica
\bibitem{Tesc} G. Teschl. Mathematical methods in quantum mechanics with applications to {Schrödinger} operators. Consultado en abril de 2005 en \url{http://www.mat.univie.ac.at/\~gerald}.

\end{thebibliography}
   % Bibliografía
}

% Descomentar en el caso de necesitar incluir apéndices
%\appendix			% Apéndices

%\chapter{CONCEPTOS PRELIMINARES}

% ---------------> conjuntos, funciones y sucesiones
\section{subespacios, espacio producto, espacios de unión disjuntos}
Esta es una nueva linea

\par}               % termina interlineado 1 1/2

\end{document}
