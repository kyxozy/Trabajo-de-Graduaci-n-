\chapter{CONCEPTOS PRELIMINARES}

\section{subespacios, espacio cociente}

\subsection{Subespacios}
Sea $\mathrm{X}$ un espacio topológico y sea $ \mathrm{S} \, \subseteq \, \mathrm{X}$ un subconjunto. Entonces definimos $\tau_{s}$ en $\mathrm{S}$ como 
\begin{equation*}
\tau_{s} = \{ \mathrm{U} \subseteq \mathrm{S}: \mathrm{S} \cap \mathrm{V} \,\,  para \,\, algun \,\,  subconjunto \,\, abierto \,\, \mathrm{V} \subseteq \mathrm{X} \}
\end{equation*}

\begin{prp} Supongamos que $\mathrm{S}$ es subconjuto de un espacio topológico $\mathrm{X}$.
	\begin{enumerate}
		\item Si $\mathrm{U} \subseteq \mathrm{S} \subseteq \mathring{X}$, $\mathrm{U}$ es un abierto en $\mathrm{S}$, y $\mathrm{S}$ es abierto en $\mathrm{X}$, entonces es $\mathrm{U}$ es abierto en $\mathrm{X}$, lo mismo para cerrados.
		\item Si $\mathrm{U}$ es un subconjunto de $\mathrm{S}$ que es abierto y cerrado en $\mathrm{X}$ entonces también es abierto y cerrado en $\mathrm{S}$.
	\end{enumerate}

\end{prp}

\subsection{Espacio Cociente}
Supongamos que $\mathrm{X}$ es un espacio topológico, y supongamos que tenemos una relación de equivalencia definido en $\mathrm{X}$. Sea $\mathrm{X}^{*}$ el conjunto de $\sim$, entonces queremos definir una topología en $\mathrm{X}^{*}$, la cual es llamada topología cociente. 
Para ello tomaremos una función (proyección canónica). 
\begin{equation*}
\pi : \mathrm{X} \rightarrow \mathrm{X^{*}} 
\end{equation*}
la cual esta definida por
\begin{equation*}
\pi (x) =[x]
\end{equation*}
Es decir $\pi$ es la función de $\mathrm{X}$ en el conjunto potencia de $\mathrm{X}$ que asigna a cada $x \,  \in \, \mathrm{X}$ a un subconjunto de $\mathrm{X}$
llamado las clases de equivalencia del punto $x$.
Debido a que  $x \,  \in \, \mathrm{X}$ esta en una sola clase de equivalencia, la función $x \rightarrow [x] $ esta bien definida, ya que cada clase de equivalencia tiene al menos un elemento, esta función es sobreyectiva.  
Algunos ejemplos de conjuntos cocientes $\mathrm{X}^{*}$

\begin{exa}
wewf
\end{exa}


\section{Topología cociente}
Pensando en la construcción del espacio $\mathrm{X}^{*}$ tomando un conjunto de $\mathrm{X}$ tomando algunas partes juntas, queremos que la transición de $\mathrm{X}$ a $\mathrm{X}^{*}$ sea continua. Tomando la proyección como 

\begin{equation*}
\pi : \mathrm{X} \rightarrow \mathrm{X}^{*} \,\,\,\,\,\,\,\,\,\,\,\,\, \pi(x) [x]
\end{equation*}

continua, con esto obtenemos que en $\mathrm{X}^{*}$ : si un conjunto $\mathrm{U}$ es abierto en $\mathrm{X}^{*}$  entonces $\pi^{-1}(\mathrm{U})$ es abierto en $\mathrm{X}$ 



